\chapter{实验结果与分析}
\label{cha:usage-example}

\section{实现细节} \label{sec:implementation-details}
\textbf{数据分割:}
数据集被分为训练和测试集,比例为4:1,分别包含52k和13k的图像-文本对。本文进一步开发了一个用于人类评估的测试分割(参见~\ref{sec:human-evaluation}节),其中提示来自HumanArt~\cite{ju2023humanART},每个提示描述了人类执行困难动作,如杂技、舞蹈和戏剧。

\textbf{模型和训练设置:}
\begin{itemize}
    \item 异常评分器:根据Laion~\cite{schuhmann2022laion}中概述的美学评估方法,本文开发了一个异常评分器。这个模型基于多层感知机(MLP)的架构,并建立在预训练的CLIP视觉变换器大型(ViT-L)模型~\cite{CLIP}的基础上。这种设计选择允许对图像数据中的异常进行鲁棒和细致的评估。
    \item 异常检测器:本文在训练集上对YOLOv8 \cite{yolo_redmon2016you} 和 RT-DETR \cite{lv2023detrs}进行了100个周期的微调。随后,本文在测试集上进行相应的检测,获得了在异常肢体检测上一致稳定的结果。
    \item 负面提示微调:本文对SDXL~\cite{SDXL}发布的检查点在1024 x 1024的分辨率上进行了10个周期的微调,学习率为1.0e-5,在训练分割上,总批量大小为32。图像与原始文本描述配对,后跟异常标题(例如,Ab.hand, Ab.leg)。
\end{itemize}

\subsection{异常检测器}
本文将数据集分为训练集和测试集,比例为4:1。随后,本文对YOLOv8~\cite{yolo_redmon2016you}和RT-DETR~\cite{lv2023detrs}进行了微调,以作为本文的检测器。用于微调YOLOv8的模型参数详见表~\ref{tab_detection_para}。在测试集上获得的结果总结在表~\ref{tab_detection_res}中。其中,YOLOv8n和YOLOv8x是YOLOv8系列中的两个模型,YOLOv8x包含更多的层和参数。在追求提高检测性能的过程中,本文选择YOLOv8x作为本文流程中的主要检测器。


\begin{table}[h]
\footnotesize
    \centering
    \caption{YOLOv8模型的参数设置。}
    \begin{tabular}{c|ccccccccc|cccccccc}
    \hline
    \textbf{参数} & \textbf{nc} & \textbf{batch} & \textbf{patience} & \textbf{pre-trained} & \textbf{lr0} & \textbf{weight\_decay} & \textbf{NMS} & \textbf{iou} \\
     \textbf{值} & 18 & 32 & 50 & true & 0.01 & 0.0005 & false &0.7  \\
     \hline
     \textbf{参数} &\textbf{epochs} & \textbf{images} & \textbf{workers} & \textbf{optimizer} & \textbf{momentum} & \textbf{augment} & \textbf{conf} & \textbf{max\_det} \\
      \textbf{值} & 120 & 640 & 8 & auto & 0.937 & false & null & 300 \\
    \hline
    \end{tabular}
    \label{tab_detection_para}
\end{table}

\begin{table}[h]
  \centering
  \caption{AbHuman数据集测试集上异常肢体检测结果。}
    \begin{tabular}{cccc}
    \toprule
    模型 & Yolov8n & Yolov8x & RT-DETR \\
    \midrule
    mAP50 $\uparrow$ & 0.333 & \textbf{0.426} & 0.331 \\
    mAP50-95 $\uparrow$ & \textbf{0.282}  & 0.235  & 0.188 \\
    \bottomrule
    \end{tabular}%
  \label{tab_detection_res}%
\end{table}%

\subsection{异常评分器}


\textbf{训练细节:} 本文使用学习率1.0e-4,通过AdamW~\cite{loshchilov2017decoupled}优化器,在AbHuman训练分割上对异常评分器进行了30个周期的微调,批量大小为256。被分配了异常标签的图像为正样本,而正常的为负样本。最后的线性层后跟一个sigmoid层,使用二元交叉熵作为目标函数。

异常肢体检测的微调结果展示在表 \ref{tab_detection_res} 中,一些可视化的检测结果展示在图 \ref{fig_detection_example} 中。YOLOv8 \cite{yolo_redmon2016you} 和 RT-DETR \cite{lv2023detrs} 都一致地达到了稳定的结果。然而,由于人类姿势和场景风格的多样性,微调后的检测结果低于原始模型在MS-COCO\cite{MSCOCO_lin2014microsoft}上的预测。

\section{定量实验} \label{sec:exps-quantitative}



表\ref{tab:quantitative}展示了本文的HumanRefiner与其他几种模型在处理AbHuman数据集上的表现。可以看到,本文的模型在异常分数、CLIP分数和FID分数上均优于其他模型,表明HumanRefiner在减少异常、保持内容相关性以及提高图像质量方面具有较强的能力。这些结果证实了本文方法的有效性,尤其是在处理复杂、多样化的人类姿态和场景时,能够有效地改善生成图像的质量和真实性。

\begin{table*}[h]
\centering
\caption{在AbHuman完整测试集(来自LAION的提示)和困难测试集(来自HumanArt的提示)上的异常分数、CLIP分数和FID分数。}
\resizebox{1\columnwidth}{!}{
\begin{tabular}{c|ccc|ccc}
\hline
数据集 & \multicolumn{3}{c|}{LAION} & \multicolumn{3}{c}{HumanArt}\\
\hline
模型              & 异常分数$\downarrow$ & CLIP分数$\uparrow$& FID分数$\downarrow$ & 异常分数$\downarrow$ & CLIP分数$\uparrow$ &  FID分数$\downarrow$ \\ \hline
SSD-1B\cite{segmind_ssd1b}  & 0.612 & 34.02& 16.465 & 0.832 & {33.37} & 13.304 \\
PixArt-XL-2-1024-MS\cite{chen2023pixartalpha}  & 0.701 & 34.32& 19.604 & 0.807 & 32.68 & 15.452\\
DeepFloyd-IF\cite{shonenkov2023deepfloyd} & 0.595 & 32.72& 25.926 & 0.849 & 32.67 &21.512\\
LCM\cite{luo2023latent}  & 0.644 & 32.99& 25.175 & 0.831 & 32.40 & 21.179\\
SDXL\cite{podell2023sdxl} & 0.659 & 34.13& 30.024 & 0.857 & 32.40 & 20.796\\
Pose-ControlNet\cite{zhang2023adding} & 0.663 & 33.51& 15.368 & 0.838 & 31.88 & 13.047\\
Pose-T2I-Adapter\cite{mou2023t2i} & 0.624 &  34.79 & 15.658 & 0.807 & 32.72 &12.582\\
HumanSD\cite{ju2023humansd} & 0.661 & 33.78& 50.546 & 0.801 & 33.03 & 38.032\\
HumanRefiner(本文的) & \textbf{0.590} & \textbf{34.85}& \textbf{13.634} & \textbf{0.778} & \textbf{33.90} &\textbf{9.145}\\
\bottomrule
\end{tabular}}
\label{tab:quantitative}
\end{table*}



\textbf{评估设置:} 本文与包括SOTA文本到图像扩散模型SDXL~\cite{SDXL},DeepFlyod-IF~\cite{shonenkov2023deepfloyd},LCM~\cite{luo2023latent}在内的基线方法进行了定量比较。对于需要额外姿态条件输入的方法,如HumanSD~\cite{ju2023humansd},Pose-ControlNet~\cite{controlnet}和Pose-T2I-Adapter~\cite{t2i_adapter},本文首先从SDXL~\cite{SDXL}生成一个初始图像,然后用姿态检测器~\cite{openpose}从中生成姿态条件,这与本文的姿态引导生成操作类似。测试提示来自AbHuman完整测试分割(来自LAION的提示)和困难测试分割(来自HumanArt的提示)。本文分别用异常分数、CLIP分数和FID定量评估生成的图像。异常分数衡量异常的严重程度。CLIP分数通过计算输入文本和生成图像之间的特征距离来评估图像-文本对齐。FID在生成的图像和LAION或HumanArt的GT图像之间计算。

\textbf{结果:} 如表~\ref{tab:quantitative}所示,HumanRefiner在异常分数、FID和CLIP分数方面始终达到最佳性能。此外,与在SDXL上应用的Pose-ControlNet~\cite{controlnet}和Pose-T2I-Adapter~\cite{t2i_adapter}相比,HumanRefiner展示了显著的改进,表明HumanRefiner在集成现成方法(如姿态检测器和T2I-adapter~\cite{t2i_adapter})方面的有效性。

\label{sec:human-evaluation}
\begin{figure*}[htb]
    \centering
    \includegraphics[width=1\columnwidth]{fig/human-evaluation1.pdf}
    \caption{基于HumanArt提示的人类评估。百分比表示人类偏好的频率。"Tie"意味着具有相似偏好水平。}
    \label{fig:human evaluation}
\end{figure*}


\textbf{人类评估:} 本文对使用HumanArt~\cite{ju2023humanART}提示生成的图像进行了人类偏好研究,询问了38名用户在DALL-E 3~\cite{openai2023dalle3}、SDXL~\cite{SDXL}和提出的HumanRefiner之间进行偏好比较。每位用户分别从真实性和肢体质量两个方面对图像进行比较。用户也可以选择“平局”,意味着两幅图像处于同一水平。在整个评估过程中,用户不知道图像是由哪个模型生成的。图~\ref{fig:human evaluation} 显示,HumanRefiner在人类生成的真实性和肢体质量方面均超过了DALL-E 3和SDXL。在肢体质量方面,HumanRefiner获得了比SDXL高2.9倍的偏好(32.3\% 对比 11.20\%),以及比DALL-E 3高1.4倍的偏好(51.42\% 对比 36.03\%),表明HumanRefiner能够生成高质量的人类姿态。

\textbf{推理速度:}HumanRefiner在单个V100 GPU上运行效率高。表~\ref{tab:infer-speed}比较了HumanRefiner与三个基线在生成单个1024 x 1024图像的推理时间。尽管HumanRefiner相比于SDXL和HumanSD需要额外的时间,但它仍然比引入额外网络模块用于人类生成的Pose-ControlNet有更好的效率。

\begin{table*}[!h]
\centering
\caption{HumanRefiner与三个基线生成单个1024 x 1024图像的推理时间}
\begin{tabular}{l|cccc}
\toprule
方法 & SDXL & Pose-ControlNet & HumanSD & HumanRefiner \\ 
% \hline
\midrule
时间成本 (秒) & 15.96 & 36.03 & 9.39 & 28.13 \\ 
\bottomrule
\end{tabular}
\vspace{-2mm}
\label{tab:infer-speed}
\end{table*}



\begin{table*}[h!]
\centering
\caption{对提出的组件进行消融研究,在AbHuman测试集(来自Laion的提示)上使用异常分数进行评估。}
\vspace{-2mm}
\begin{tabular}{cccc|c}
\toprule
负面提示 & 异常引导 & 检测器修补 & 姿态引导 &    异常分数 $\downarrow$ \\
\midrule
\cxmark & \cxmark & \cxmark & \cxmark & 0.659 \\
\checkmark & \cxmark & \cxmark & \cxmark & 0.614 \\
\checkmark & \checkmark & \cxmark & \cxmark & 0.605 \\
\checkmark & \checkmark & \checkmark & \cxmark & 0.601 \\
\checkmark & \checkmark & \checkmark & \checkmark & \textbf{0.590} \\
\bottomrule
\end{tabular}
\label{tab:ablation}
\end{table*}

\textbf{消融研究:}
本文进行了组件分析,以评估每个元素的贡献,使用异常分数作为生成质量的度量。如表~\ref{tab:ablation}所示,每个组件对异常分数都有明显的贡献。例如,从AbHuman数据集学习到的负面提示的引入显著降低了异常分数,从0.659降低到了0.614。

为了更清晰地展示HumanRefiner的修正成果,本文在图~\ref{fig:ablation_examples}中展示了几个消融实验的发现。\textbf{基线}代表SDXL模型未经修改的输出,而\textbf{异常引导}和\textbf{姿态引导}展示了通过应用不同引导模块实现的结果。如图中第一行所示,在人物轮廓上观察到了明显的增强。此外,第二行和第四行展示了手部细节改善的不同程度。值得注意的是,即使在手部细节通常显得模糊的场景中,本文的\textbf{异常引导}模块也展示了修正的效果。重要的是,本文的\textbf{异常引导}模块的消融类似于数据集的消融,强调了数据集在本文方法中的关键作用。

\section{定性结果}
本文还在困难测试分割中展示了示例,包括参与舞蹈或表演杂技等活动的人类。如图~\ref{fig:visualize-compare}所示,与基线方法相比,HumanRefiner展现了更优越的肢体质量。它准确捕捉了人体姿势,呈现了自然的肢体属性,并展现了视觉上令人愉悦的布局。
\newpage

\begin{figure}[t]
    \centering
\includegraphics[width=0.9\linewidth]{fig/visualize-compare.pdf}
   \caption{在参与活动提示的人类上进行的与基线的定性比较。}
    \label{fig:visualize-compare}
    \vspace{-5mm}
\end{figure}

\begin{figure}[b]
    \centering
    \includegraphics[width=0.8\columnwidth]{fig/ablation_examples1.pdf}
    \caption{消融实验的可视化结果。}
    \label{fig:ablation_examples}
\end{figure}

\section{HumanRefiner的更多可视化}
本节展示了本文的HumanRefiner流程的中间结果,包括异常引导的输出、检测到的姿态、姿态引导输出以及修补的输出,见图~\ref{fig:visualize-examples-1}。



\begin{figure*}[htb]
	\centering  %图片全局居中
	\subfloat[运动, 一位穿着休闲装的年轻BMX骑手在滑板公园中腾空而起]{
		\includegraphics[width=\linewidth]{./fig/316.png}} \\
	\subfloat[战斗, 一位为彩弹比赛装备好的玩家正在森林中前进]{
		\includegraphics[width=\linewidth]{./fig/368.png}} \\
	\subfloat[瑜伽, 两个人在日落时分的沙滩上表演瑜伽姿势]{
		\includegraphics[width=\linewidth]{./fig/370.png}} \\
	\subfloat[舞蹈, 一位穿着白色连衣裙的芭蕾舞者站在墙前]{
		\includegraphics[width=\linewidth]{./fig/ap4.png}}
	\caption{HumanRefiner流程的中间输出,从左到右:(1) 异常引导,(2) 姿态检测,(3) 姿态引导的粗略修正,和 (4) 修补输出。用\textcolor{red}{红色}框突出所有异常部分,并用\textcolor{green}{绿色}框标记相应的修正后的正常部分。 }
     \label{fig:visualize-examples-1}
\end{figure*}