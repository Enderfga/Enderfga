%% chapter 5 dataset, network structure, experiment and result
\chapter{结论与未来工作}
本章全面梳理了本文的研究成果,重点突显了\textbf{HumanRefiner}算法在人体异常修正任务中的革新之处与技术意义。具体而言,HumanRefiner作为一款针对异常肢体的细粒度注解基准AbHuman上开发的即插即用模块,成功实现了与各类文本到图像扩散模型的无缝整合。其核心技术创新体现在整合了异常引导、可逆姿态引导及检测器引导的多维度修复机制,确保在图像生成过程中准确识别并剔除异常肢体,显著提升了合成人类图像的品质。

通过对AbHuman测试集进行严谨的定量评估及获得积极的人类评估反馈,证实了HumanRefiner相较于现有方法的优越性能,彰显了其对文本到图像生成领域,特别是在人体图像修复研究方面的实质性贡献。然而,尽管HumanRefiner在高质量人类图像生成方面已取得突破,当前研究仍存在以下两个主要局限性:

1. \textbf{推理速度受限}:由于涉及双重生成步骤与修补操作,HumanRefiner的实时性能或大规模应用的效率受到一定影响,有待通过算法优化提升其运行速度。

2. \textbf{数据集场景覆盖不全}:尽管AbHuman在异常肢体注释方面详实,但对于正常人类图像在不同场景下的注释范围相对有限,可能制约模型在新场景下的泛化能力。

基于以上现状与未来技术发展趋势,我们前瞻性地规划了以下几个关键研究方向:

1. \textbf{提升文本指导精度与灵活性}:深化研究如何强化文本指令与图像生成之间的关联性,提高模型对复杂、精细文本描述的理解与执行能力,以实现更精确的人体图像定制。

2. \textbf{开发轻量化模型与加速策略}:探索架构简化、模型压缩及高效计算技术,力求在保持HumanRefiner修正效果的同时,大幅降低其计算复杂度与运行时间,提升实际应用的可行性。

3. \textbf{结合3D技术提升真实感}:研究整合3D建模与渲染技术,利用深度信息和立体感知增强生成人体图像的立体感与物理真实性,以逼近甚至超越真实照片级别的视觉效果。

这些前瞻性的研究展望旨在为后续学术研究和技术开发提供有价值的启示,驱动文本到图像生成技术在人体图像修复领域的持续创新与性能升级,最终助力相关技术在虚拟现实、影视制作、游戏开发等多个应用场景中的广泛应用。