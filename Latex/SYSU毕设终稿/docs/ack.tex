%%
% 致谢
% 谢辞应以简短的文字对课题研究与论文撰写过程中曾直接给予帮助的人员(例如指导教师、答疑教师及其他人员)表示对自己的谢意,这不仅是一种礼貌,也是对他人劳动的尊重,是治学者应当遵循的学术规范。内容限一页。
% modifier: 黄俊杰
% update date: 2017-04-15
%%

\chapter{致谢}

时间如同白驹过隙,转瞬间,四年的本科生活即将画上圆满的句号。这段时光,充满了青春的活力与甜美,如同一幅精致的画卷,绘着我成长的点滴。在这四年中,我不仅沉浸于知识的海洋,汲取着学问的养分,还在挑战与机遇中磨砺了自我,提升了分析问题和解决问题的能力,同时也培养了扎实的科学素养。虽然这段旅程并非一帆风顺,我曾在探索的道路上迷失方向,但正是这些经历让我对学术研究的道路充满了更加坚定的信念与热爱,每一步都显得格外珍贵。所有的这些收获,都离不开我敬爱的老师们的悉心教导和亲爱的朋友们的无私帮助,我在此想向他们表达我最深切的感激之情。

特别地,我要向我的指导老师——梁小丹副教授——表达我最深的敬意。作为一名尚处学术起步阶段的本科生,我在学术研究的道路上还显得稚嫩和迷茫,常常难以把握研究问题的关键和难点,对于自己的工作能达到何种水平也心存疑虑。在这样的背景下,梁老师用她对领域的深刻理解、丰富的知识和独到的洞察力,为我指明了前进的方向,给予了宝贵的指导。她那严谨的学术态度和勤勉的工作精神,更是成为我学习的典范。在此,我想对梁老师表达最崇高的敬意和最衷心的感谢。

我还要向我的家人致以最深的谢意。是他们无私的奉献和全力的支持,为我提供了坚实的后盾,让我在学术路上勇往直前,无所畏惧。正是因为有了他们的鼓励和陪伴,我才能在这条充满挑战的道路上取得今天的成绩,他们的爱是我最宝贵的财富。

随着本科生活的结束,新的旅程即将开启。我将带着这份深情的感激,继续前行,在未来的学术探索中不断超越自我,为梦想插上翅膀。
\vskip 108pt
\begin{flushright}
	方桂安\makebox[1cm]{} \\
	\today
\end{flushright}

